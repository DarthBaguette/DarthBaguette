\documentclass[../PhysicsFormulae.tex]{subfiles}
\begin{document}

\subsection{Constant Acceleration}
For constant acceleration, motion is governed by
\[\Delta x = v_{avg}t\]
\[\Delta x = v_0 + \frac{1}{2}at^2 = v - \frac{1}{2}at^2\]
\[v=v_0+at\]
\[v^2 = v_0^2 + 2a \Delta x\]

\subsection{Variable Acceleration}
More generally, though, the relation between velocity and acceleration is given by
\[\vec{v}(t) = \vec{v}(t_0) + \int_{t_0}^{t} \vec{a}(t) \,dt \]

\subsection{Projectile Motion}
Consider a particle launched from ground level with velocity $\vec{v}$ at angle $\phi$. Then, the speed at any point is given by $v = \sqrt{v_x^2+v_y^2}$ and angle to the horizontal of $\phi = \tan^{-1}(v_y/v_x)$. \bigskip

The particle's shape is (approximately, assuming $\vec{g}$ is constant) a parabola given by
\[y = (\tan{\phi})x - \left(\cfrac{g}{2v_x^2}\right)x^2\]

The range of a particle is given by
\[R = \cfrac{v_0^2 \sin(2\phi_0)}{g}\]

The maximum height of a particle is given by
\[h = \cfrac{v_0^2 \sin^2{\phi_0}}{2g}\]

\subsection{Kinematics in Polar Coordinates}
The unit vectors in polar coordinates are not constant, given by 
\[ \hat{r} = \cos{\theta} \hat{i} \sin{\theta} \hat{j} \]
\[ \hat{\theta} = -\sin{\theta} \hat{i} + \cos{\theta} \hat{j} \]
The derivatives of these unit vectors are 
\[ \frac{d\hat{r}}{dt} = \dot{\theta} \hat{\theta} \]
\[ \frac{d\hat{ \theta} }{dt} = -\dot{\theta} \hat{r} \]
Therefore, letting a particle's position be 
\[ \vec{r} = r \hat{r} \]
we obtain the velocity and acceleration in polar coordinates, 
\[ \vec{v} = \frac{d\vec{r}}{dt} = \dot{r} \hat{r} + r \dot{\theta} \hat{\theta} \]
\[ \vec{a} = (\ddot{r} - r\dot{\theta}^2) \hat{r} + (r\ddot{\theta} + 2\dot{r}\dot{\theta} ) \hat{\theta} \]
Here's the intuition for the acceleration equation---consider, what does each term represent?
\begin{itemize}
	\itemsep0em
	\item Case 1: Radial acceleration. The first term, $\ddot{r}$, refers to the change in radial velocity, while the second term is the centripetal acceleration of the object (which is always radially inwards, hence the negative sign). 
	\item Case 2: Tangential acceleration. The first term, $r \ddot{\theta}$, comes from the change in tangential speed. The second term is known as the Coriolis acceleration, which, contrary to the fictitious nature of the Coriolis force, is actually a very real acceleration. 
\end{itemize}

\subsubsection{Translational-Rotational Vector Relationships}
By considering circular motion, we can obtain some general relations between vectors:
\[ \vec{v} = \vec{\omega} \times \vec{R} \]
\[ \vec{a}_T = \vec{\alpha} \times \vec{R} \]
\[ \vec{a}_R = \vec{\omega} \times \vec{v} = \vec{\omega} \times (\vec{\omega} \times \vec{R} ) \]

\end{document}