\documentclass[../PhysicsFormulae.tex]{subfiles}

\begin{document}

\subsection{Types of Materials}
The conductivity of a material correponds the number of \textit{conduction electrons} it has. 
\begin{itemize}
	\itemsep0em
	\item In \textbf{conductors}, each atom may give up one or two electrons
	\item In \textbf{semiconductors}, 1 in $10^10$ to $10^12$ atoms may give up an electron
	\item In \textbf{insulators}, electrons are not free to move
	\item In \textbf{superconductors}, there is absolutely no resistance to the flow of electrons
\end{itemize}
However, insulators can behave more like conductors if a strong enough electric field is applied (known as \textit{breakdown}). \bigskip 

\textit{Polar} materials have permanent dipole moments. Electric fields can align the dipoles of molecules; in \textit{ferroelectric} materials, the dpioles stay aligned even after the field is removed. 

\subsection{Conductor in an Electric Field}
If an electric field is applied to a conductor, \textit{current} may begin to flow:
\[ i = \frac{dq}{dt} \]
with units of amperes. While it has both magnitude and direction, current is not a vector... but current density is!
\[ j = \frac{i}{A} \rightarrow i = \int \vec{j} \cdot d\vec{A} \]

\subsection{Current Density and Drift Speed}
When current flows, electrons drift through at a constant speed, even in the presence of an electric field, because they constantly crash into the lattice of ions and lose energy. Consider current moving through a segment of a conductor of electron density $n$ and length $L$, 
\[ j = \frac{q}{At} = \frac{enAL}{A(L/v_d)} = env_d \]
In vector form, 
\[ \vec{j} = -en\vec{v}_d \]

\subsection{Ohmic Materials}
Because electric field and current density are both proportional to the drift velocity, it stands to reason that they are proportional to each other:
\[ \vec{j} = \sigma \vec{E} \longleftrightarrow \vec{E} = \rho \vec{j} \]
where $\sigma \; [\mathrm{\frac{S}{m}}]$ (siemens per meter) is the conductivity and $\rho \; [\Omega m]$ (Ohm meters) is the resistivity. The siemen and ohm are defined as
\[ 1 \; \mathrm{siemen} = 1 \; \mathrm{\frac{ampere}{volt}} \]
\[ 1 \; \mathrm{ohm} = 1 \; \mathrm{\frac{volt}{ampere}} \]
In some materials, the resistivity varies with the applied electric field. But in \textbf{Ohmic} materials, the resistivity is independent of the external field. Many homogeneous materials (including conducting metals) obey Ohm's Law for a range of strengths of the applied field. \bigskip

Rearranging the definition of resistivity,
\[ \rho = \frac{E}{j} = \frac{\Delta V/L}{i/A} = \frac{\Delta V}{i} \frac{A}{L} \]
The resistance is defined as $R = \Delta V / i$, or
\[ R = \rho \frac{L}{A} \]
Thus, we can rephrase Ohm's Law as: the resistance of an object is independent of its applied electric field. While ordinary resistors are ohmic, semiconducting devices such as diodes and transistors tend not to be. \bigskip

In general, finding the resistance of weird objects often involves applying an electric field across it. Then, $\Delta V = \int \vec{E} \cdot d\vec{s}$ and $i = \int \vec{j} \cdot d\vec{A}$ can be measured, allowing resistance $R$ to be solved for. 

\subsection{Current and Heat Analogy}
We can rewrite the definition of resistance as
\[ i = \frac{\Delta V}{R} = \frac{\Delta V}{\rho \Delta x/A} = \sigma A \frac{\Delta V}{\Delta x} \]
In differential form, 
\[ \frac{dq}{dt} = -\sigma A\frac{dV}{dx} \]
which is analagous to the heat flow equation, 
\[ \frac{dQ}{dt} = -kA\frac{dT}{dx}\]
Pure metals share more than a nice mathematical analogy; both heat and electricity are conducted via their free electrons. 

\subsection{Temperature Variance of Resistivity}
The relation between resistivity and temperature is linear, 
\[ \Delta \rho = \rho_0 \alpha_{av} \Delta T \]
Rearranging and taking the limit as the two points converge, 
\[ \alpha = \frac{1}{\rho}\frac{d\rho}{dT} \]
The value $\alpha$ is known as the temperature coefficient of resistivity. 

\subsection{Ohm's Law: A Microscopic View}
The drift speed in a conducting material is
\[ v_d = at = -\frac{eE\tau}{m} \]
where $\tau = \lambda / v_{av}$ is the mean time between collisions. But $v_d = E/\rho en$ too, so we can rearrange as
\[ \rho = \frac{m}{ne^2 \tau} \]
Thus, conductivity is proportional to the electron density, which checks out intuitively. 

\subsection{Insulators in Electric Fields}
Consider an electric field $E_0$ applied to an insulating material. The polarization of atoms will lead to an induced field $E'$ that is proportional to the applied field, so the net field is 
$E = E_0 - E' = \frac{1}{\kappa} E_0$
where $\kappa$ is the dielectric constant. Insulating materials are also known as \textit{dielectric materials}, and the electric field at which they break down are known as \textit{dielectric strengths}. \bigskip

Because electric fields are inversely proportional to both $\epsilon_0$ and $\kappa$, we define the \textit{permittivty} of a material as 
\[ \epsilon = \kappa \epsilon_0 \]
and thus, $\epsilon_0$ is known as the permittivity of free space. 



\end{document}