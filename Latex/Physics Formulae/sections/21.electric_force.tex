\documentclass[../PhysicsFormulae.tex]{subfiles}
\begin{document}

\subsection{Electric Charge}
Charge is a scalar quantity created by the transfer of electrons. However, it is a \textit{derived unit}, defined as the ampere-second because it is very difficult to measure the force between two static charges. 

\subsection{Atomic Structure}
Electric charge is quantized, meaning that all charges are multiples of the elementary charge $e$: 
\[ e = 1.602 \times 10^{-19} \; \mathrm{C} \] 
which means that number of electrons per Coulomb is  
\[ N = 6.242 \times 10^{18} \] 

All particles such as electrons and protons have \textit{antiparticles}, which have the same mass but opposite charge. Specifically, they are the positron and antiproton for these ones. \bigskip 

Ordinary nuclei have $Z$ protons, known as the atomic number, so that the charge of the nucleus is $+Ze$. Electrons make the atom neutral overall, but ions can be formed by adding and removing electrons. \bigskip 

While electrons are \textit{fundamental} particles since they have no substructure, protons and neutrons are believed to be made up of more elementary entities known as quarks. An up quark has $q = +\frac{2}{3}e$ and a down quark has $q = -\frac{1}{3}e$, so a proton has 2 up quarks and 1 down quark, while a neutron has 1 up quark and 2 down quarks. 

\subsection{Conductors and Insulators}
\textit{Conductors} are materials in which electrons can move relatively freely; their electrons do not belong to specific atoms but rather flow readily. Examples include tap water and the human body. \textit{Insulators}, contrarily, are materials in which electrons can hardly flow at all. Examples include glass, plastics, and many crystalline materials such as $\mathrm{NaCl}$. \bigskip 

When an object has a source from which to obtain or expel electrons, it is known as being \textit{grounded}, and as a result there is no net charge. The Earth is a common source and is considered to possess an infinite supply of electrons. \bigskip 

Isolated atoms of conducting materials have loosely bound electrons known as \textit{conduction electrons}. They can easily move through the material, leaving fixed positive ions in their wake. \bigskip

\textit{Polarization} allows insulators to be affected by electric forces, too. Consider a group of electrons near a block of wood; the electrons in the wood are slightly repelled and the protons are slightly attracted, creating a net charge gradient in the wood. 

\subsection{Charging by Contact and Induction}
Consider a neutral copper rod put in contact with a positively charged glass rod. Electrons will move from copper to glass in order to neutralize the glass, but only at the point of contact. By wiping the electrons to other areas, the entire rod can be neutralized. This is \textit{charging by contact}. \bigskip-

Now consider the polarized copper rod in the presence of the positive glass, connected to the ground by a string. Electrons flow from the ground to the positive end of the copper to neutralize it, and if we disconnect the string, the copper rod thus retains net negative charge. This is \textit{charging by induction}. 

\subsection{Coulomb's Law}
\textbf{Coulomb's Law} gives the force between two point charges, 
\[ F = \frac{1}{4\pi \epsilon_0}\frac{q_1 q_2}{r^2} \]
where $\epsilon_0 = 8.854 \times 10^{-12} \; \mathrm{\frac{C^2}{Nm^2}}$. In vector form, this becomes
\[ \vec{F}_{12} = \frac{1}{4\pi \epsilon_0}\frac{q_1 q_2}{r^2}\hat{r}_{12} \]
where $12$ denotes the force on 1 by 2 and the direction to 1 from 2. 

By breaking up distributions of charge into small elements that are 'point charges', electric forces can be found. 

\subsection{Shells of Charge}
Consider a thin spherical shell of charge. We have 
\begin{enumerate}
	\item There is no electric force exerted on a point charge located anywhere inside the shell. 
	\item The shell can be treated as a point charge regarding forces on other charges outside the shell.
\end{enumerate}

\subsection{Conservation of Charge}
No exceptions have ever been found to conservation of charge. Particles at the subatomic level must follow these rules. 
\begin{itemize}
	\item Hydrogen ion: $\mathrm{H}$ (one proton)
	\item Deuterium: $\mathrm{^2 H}$ (one proton, one neutron); `heavy hydrogen'
\end{itemize}
\end{document}