\documentclass[../PhysicsFormulae.tex]{subfiles}
\begin{document}
\subsection{Brownian Motion}
\subsection{Brownian Motion}
Brownian motion is characterized by the erratic movement of molecules at an atomic level. A particle suspended in a fluid is bombarded on all sides by atoms of the fluid; on average, it will experience no net movement, but due to randomness there will be fluctuations around its original position, given by
\[ [(\Delta x)^2]_{av} = \frac{RT}{3\pi\eta aN_A}\Delta t \]
if the bombarded particle is of radius $a$. Known as \textit{random walk} patterns, these motions supported the atomic theory of matter. 

\subsection{Properties of Ideal Gases}
\begin{enumerate}
	\itemsep0em 
	\item Large number of molecules---makes the mean more constant
	\item Molecules are of negligible volume---implies that the volume of the container is the free volume that the gas can move through
	\item No intermolecular forces---electromagnetic, quantum, etc.
	\item Collisions are elastic and of negligible duration---implies kinetic energy is constant and potential energy of molecules is negligible
\end{enumerate}

\subsection{Molecular View of Pressure}
Consider one particle bouncing off one wall of a cuboidal box. The force delivered to the particle is 
\[ F_x = \frac{2mv_x}{2L/v_x} \]
Extending this reasoning to all particles, 
\[ p = \frac{F_x}{A} = \frac{Nm}{L^3}\frac{(v_{1x}^2 + v_{2x}^2 + \cdots + v_{Nx}^2)}{N} \]
Then,
\[ p = \frac{1}{3}\rho (v^2)_{av} \]
because $v^2 = v_1^2 + v_2^2 + v_3^2$, with no preference among the three directions. 
Finally, solving for the root-mean-square speed yields
\[ v_{rms} = \frac{3p}{\rho} \]

\subsection{Mean Free Path}
The mean free path is the average distance a molecule travels between collisions. Consider if a molecule doubles in radius while all others become particles. Then, as the particle travels through a cylindrical volume in time $t$, 
\[ \lambda = \frac{L_{cyl}}{N_{cyl}} = \cfrac{v_{av}t}{N \cdot \frac{\pi d^2 v_{rel}t}{V}} = \frac{kT}{\pi d^2p} \]
The relative speed of two molecules is used in the denominator when counting the effective number of particles in its path. It can easily be seen that on average, when two particles are traveling at $90^{\circ}$ to each other, the relative speed is $v\sqrt{2}$. In real gases, the ability to conduct heat, viscosity, and rate of diffusion from high to low concentration are all proportional to the mean free path. 

\subsection{Maxwell's Speed Distribution}
Maxwell's distribution of a (large) number of molecules in a gas) gives
\[ N(v)= 4\pi N \left(\frac{m}{2\pi k T}\right)^{3/2}v^2 e^{-mv^2/2kT} \]
The product $N(v)dv$ gives the number of molecules within the speed distribution $dv$. The total number of molecules $N$ is given by 
\[ N = \int_0^{\infty} N(v)\; dv \]

\subsection{Consequences of the Speed Distribution}
\subsubsection{Most Probable Speed}
This quantity is the maximum value of N(v), given by:
\[ v_p = \sqrt{\frac{2kT}{m}} = \sqrt{\frac{2RT}{M}} \]
derived from setting $dN(v)/dv = 0$. 

\subsubsection{Average Speed}
The average speed is 
\[ v_{av} = \frac{1}{N} \int_0^{\infty} vN(v) \; dv = \sqrt{\frac{8kT}{\pi m}} = \sqrt{\frac{8RT}{\pi M}} \]

\subsubsection{Root-Mean Square Speed}
The RMS speed is the root of the average square speed, or
\[ v_{rms} = \sqrt{\frac{1}{N} \int_0^{\infty} v^2 N(v) \; dv} = \sqrt{\frac{3kT}{m}} = \sqrt{\frac{3RT}{M}} \]

\subsubsection{Average Translational Kinetic Energy}
The average translational kinetic energy per molecule is given by 
\[ {K}_{trans} = \frac{1}{2}m(v^2)_{av} = \frac{3}{2}kT \]

\subsubsection{The Ideal Gas Law}
We can use the Ideal Gas Law to rewrite the RMS speed, 
\[ v_{rms}^2 = \frac{3p}{\rho} = \frac{3RT}{M} \]
Note that the third expression comes directly from the first by substituting $\rho = \frac{\Sigma m}{V}$. 

\subsection{Experimental Verification of Maxwell Speed Distribution}
Molecules are sent from a slit in an oven through a rotating cylinder with grooves, so their speed can be found as 
\[ \frac{L}{v} = \frac{\phi}{\omega} \]
where $\phi$ is the angular displacement between the start and end of a helical groove. The rotating cylinder is thus known as a \textit{velocity selector} because only certain velocities will make it through the grooves without colliding with the walls. Adjusting this for a lot of speeds can be used to plot the Maxwell speed distribution. \bigskip

The distribution of speeds from the cylinder $N(v)$ is proportional to $v^3$, not $v^2$, because higher speeds escape the oven slit and bombard the cylinder with higher frequency, in the proportion $v_2 / v_1$. 

\subsection{Maxwell-Boltzmann Energy Distribution}
The Maxwell-Boltzmann energy distribution is applicable when molecules only have translational kinetic energy. It is based on the following argument: $N(v)dv = N(E)dE$, which yields
\[ N(E) = \frac{2N}{\sqrt{\pi}} \frac{1}{(kT)^{3/2}} E^{1/2} e^{-E/kT} \]
The factor $e^{-E/kT}$ is the \textit{Boltzmann factor} and is a general feature of the Maxwell-Boltzmann Energy Distribution, no matter the form of energy $E$. It gives a rough estimate of the relative probability for a particle to have an energy $E$ in particles at temperature $T$. \bigskip
Similary, the total number of molecules $N$ is 
\[ N = \int_0^{\infty} N(E) \; dE \]
Note that unlike the speed distribution, the mass has no effect. This is because when mass doubles, for instance, speed goes down by the square root of that factor. 

\subsection{Equations of State for Real Gases}
Real gases are not ideal at any density and depart more and more for higher densities. 

\subsubsection{Virial Expansion}
The Virial expansion adjusts the right side of the Ideal Gas Law: 
\[ pV = nRT \left[1 + B_1 \frac{n}{V} + B_2 \left(\frac{n}{V}\right)^2 + \cdots \right] \]
The constants $B_1, \; B_2, \cdots$ are called \textit{Virial coefficients} and grow successively smaller as the series progresses. As $n/V\rightarrow0$, the expansion simplifies down to the Ideal Gas Law. 

\subsubsection{Van der Waals Equation}
Van der Waals' equation of state adjusts the left side of the Ideal Gas Law: 
\[ \left(p - \frac{an^2}{V^2}\right) (V-nb) = nRT \]
\begin{enumerate}
	\item Pressure Correction: intermolecular forces are exerted, decreasing the pressure needed by walls. The force is proportional to number of molecules $n/V$, and another factor of $n/V$ because 5 times as many molecules means 5 times the correction. 
	\item Volume Correction: molecules aren't particles. Let one particle be a sphere of double radius and all other be particles; we get: 
	\[ b = \frac{V_{cor}}{n} = \cfrac{\frac{1}{2}\left(4\pi d^3\right)N}{n} = \frac{2}{3}\pi d^3 N_A \]
\end{enumerate}

\end{document}