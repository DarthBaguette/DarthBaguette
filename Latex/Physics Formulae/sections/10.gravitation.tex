\documentclass[../PhysicsFormulae.tex]{subfiles}
\begin{document}

\subsection{Newton's Law for Gravitation}
For two particles, the gravitational force is given by
\[ \vec{F}_{12} = -\frac{Gm_1m_2}{r_{12}^2} \hat{r}_{12} \]
where $\vec{F}_{12}$ is the force on 1 from 2, and $\hat{r}_{12}$ is the direction to 1 from 2.

\subsection{Shell Theorems}
There are two shell theorems that govern gravity. 
\begin{enumerate}
\item A uniform spherical shell attracts other bodies as if its mass were concentrated at its center. 
\item A uniform spherical shell has no gravitational field inside of it. 
\end{enumerate}
These are special properties of its inverse square nature.

\subsection{Gravitational Constant}
The gravitational constant is $G = 6.67 \times 10^{-11} \; \frac{Nm^2}{kg^2}$ and is relatively imprecise, as it is measured at extremely small scales with the Torsion Balance. 

\subsection{Surface Gravity}
Near Earth's surface, gravity is given by
\[ g_0 = \frac{GM_E}{R_E^2} \] 
which of course changes with latitude, as Earth's "radius" is not uniform (it bulges at the equator). The average value of $g_0$ is about $9.81 \; m/s^2$.\\
For small heights $h<<R_E$ above Earth's atmosphere, we can approximate gravity as 
\[ g = g_0 \left( 1-\frac{2h}{R_E} \right) \]

\subsection{Orbit and Escape Speed}
The orbital speed of a satellite $r$ from the Earth's center is given by
\[ v_{orb} = \sqrt{\frac{GM}{r}} \]
and the escape speed of a particle at that distance is 
\[ v_{esc} = \sqrt{\frac{2GM}{r}} \]

\subsection{Orbital Dynamics}
Orbits can be simplified as follows;
\begin{itemize}
\item Each orbiting body is only influenced by gravity from the central body; the contributions from other bodies are negligible. 
\item The central body is significantly more massive than the orbiting one, so the latter rotates purely around it. 
\end{itemize}
\subsubsection{Kepler's Laws}
Kepler experimentally determined 3 laws of orbits, given by
\begin{enumerate}
\item \textit{The Law of Orbits}: All planets move in elliptical orbits, with the Sun at a focus. 
\item \textit{The Law of Areas}: All planets sweep out equal areas in equal times. 
\item \textit{The Law of Periods}: The square of the period of revolution is proportional to the cube of the semi-major axis, given by
\end{enumerate}
\[ T^2 = \frac{4\pi^2 a^3}{GM} \]
and the constants disappear if $[T] = years$, $[a] = a.u.$, and $[G] = \frac{{a.u.}^3}{M_s y^2}$. \\
Ellipse tidbits to keep in mind: 
\begin{itemize}
\item The eccentricity is $0<e<1$ for an ellipse, and $ea$ is the distance from the center to a focus.
\item The prefix \textit{peri-} refers to least, while \textit{apo-} refers to most. Thus, perihelion is Earth's closest distance to the Sun, and aphelion is its farthest, while perigee and apogee replace them for the Moon and Earth. 
\end{itemize}

\subsubsection{Geosynchronous Orbit}
Geosynchronous orbit occurs when a satellite rotates with same period as Earth, and it occurs at altitude $r = 3.58 \times 10^7 \; m = 22,300 \; mi$ according to Kepler's 3rd Law. 

\subsection{Orbital Energies}
\subsubsection{Energy in Orbit}
The total energy of an elliptical orbit is given by
\[ E = K + U = \frac{GMm}{2a} - \frac{GMm}{a} = -\frac{GMm}{2a} \]
which implies 
\[ E = -K \]

\subsubsection{Vis-Viva}
Provided the distance and speed at another point and semi-major axis are known, as we can find the speed at any point or the point corresponding to any speed; this is derived from conservation of energy. 
\[ v^2 = GM \left( \frac{2}{r} - \frac{1}{a} \right) \]

\subsection{Orbital Types}
For an object at distance $r$ from the central body, its orbit is governed by: 
\begin{center}
\renewcommand{\arraystretch}{1.8}
\begin{tabular}{|c|c|}
\hline
Velocity & Orbital Type \\ [0.5ex]
\hline
$0<v<\sqrt{\frac{GM}{r}}$ & Elliptical; $r$ is max distance (apo-) \\
\hline
$v=\sqrt{\frac{GM}{r}}$ & Circular; $r$ is radius \\
\hline
$\sqrt{\frac{GM}{r}}<v<\sqrt{\frac{2GM}{r}}$ & Elliptical, $r$ is min distance (peri-) \\
\hline
$v=\sqrt{\frac{2GM}{r}}$ & Parabolic \\
\hline 
$v>\sqrt{\frac{2GM}{r}}$ & Hyperbolic \\
\hline
\end{tabular}
\end{center}
Note that parabolic orbits are special because they have 0 energy. 

\subsection{Gauss's Law for Gravitation}
Using the idea of gravitational fields instead of point masses, Gauss's Law yields
\[ \oint \vec{g} \cdot \,d\vec{A} = -4\pi GM_{en} \]
For an infinite line of mass: 
\[ g = \frac{2G\lambda}{r} \]
For an infinite plane of mass: 
\[ g = 2\pi G \sigma \]
For an infinite cylinder of mass: 
\[ g = 2\pi G \rho r \]
for $r<R$, and 
\[ g = \frac{2\pi G \rho R^2}{r} \]
for $r>R$. 

\end{document}