\documentclass[../PhysicsFormulae.tex]{subfiles}
\begin{document}

\subsection{Traveling Waves}
The most general formula for a wave is given by 
\[ y(x,t) = f(x-vt) \]
For a particular point on the shape of the wave to remain at constant height as it travels, we must have 
\[ x - vt = constant \rightarrow \frac{dx}{dt} = v\]
the latter of which is known as the phase velocity. $v$ always is a positive constant, so for waves traveling in the negative x direction, we have 
\[ y(x,t) = f(x+vt) \]
Ultimately, if we let $x$ be constant and vary $t$, we see how a point on the wave moves through time. If we let $t$ be constant and consider the heights at various positions $x$, we have a "snapshot" of the wave at a moment in time. 

\subsubsection{Sinusoidal Waves}
The equation for a sinusoidal wave is given by 
\[ y(x,t) = y_m\sin(kx - \omega t) \]
where $k = 2\pi / \lambda $. Moreover, the $\omega$ and $k$ are related as
\[ v = \frac{\lambda}{T} = \frac{\omega}{k} \]
The transverse speed of a particle is thus given by 
\[ \frac{\partial y}{\partial t} = - \omega y_m \cos(kx - \omega t) \]
and its acceleration as 
\[ \frac{\partial^2 y}{\partial t^2} = - \omega^2 y_m \sin(kx - \omega t) \]
In the previous equations we have assumed that $y(0,0) = 0$; the more general equation is 
\[ y(x,t) = y_m \sin(kx - \omega t - \phi) \]
where $kx - \omega t - \phi$ is the phase and $\phi$ is the phase constant, which moves a wave forwards or backwards in space or time. This can be demonstrated by writing 
\[ y(x,t) = y_m \sin\left[k\left(x-\frac{\phi}{k}\right) - \omega t\right] \]
\[ y(x,t) = y_m \sin\left[kx - \omega \left(t + \frac{\phi}{\omega}\right) \right] \]
which indicate that introducing a phase constant $\phi>0$ implies moving forwards in space or backwards in time. Both of these bring the new wave \textit{ahead} of the one with $\phi = 0$. 

\subsubsection{Wave Speed}
The speed of a wave is independent of frequency and wavelength (if it is dependent, the medium is dispersive). It is given by
\[ v = \sqrt{\frac{T}{\mu}} \]
Because it relies on small angle approximation, this only holds for relatively small transverse displacements. \bigskip

The speed in a medium is dependent on its properties, as shown above. Thus, the wavelength and frequency change inversely while speed remains constant like $v = \lambda f = \lambda ' f '$. \bigskip

However, if a medium changes, the \textit{frequency} remains constant while speed and wavelength change, as $f = v / \lambda = v' / \lambda '$.

\subsubsection{Group Speed and Dispersion}
In a real wave, which may not preserve its shape like sinusoidal waves or pulses, we must use the \textit{group speed}, which is the speed at which energy or information travels. This occurs when a medium is \textit{dispersive}: component waves (Fourier Analysis) travel at different speeds. 

\subsection{The Wave Equation}
The general form of waves on a string is given by 
\[ \frac{\partial^2 y}{\partial x^2} = \frac{1}{v^2} \frac{\partial^2 y}{\partial t^2} \]
It can be shown that $y(x,t) = f(x \pm vt)$ is a solution to this equation.\bigskip

The general form of a spherically symmetry system is given by 
\[ \frac{1}{r^2} \frac{\partial}{\partial r} \left(r^2 \frac{\partial y}{\partial r}\right) = \frac{1}{v^2} \frac{\partial^2 y}{\partial t^2} \]
which has solution 
\[ y(r,t) = \frac{A}{r} \sin(kr - \omega t) \]


\subsection{Energy in Wave Motion}
Energy is generally \textit{not} constant for a certain mass element of string: waves transmit energy in their direction of propagation. Kinetic energy at a point is given by
\[ dK = \frac{1}{2} (\mu dx) [-\omega y_m \cos(kx - \omega t)]^2 \]
and thus 
\[ \frac{dK}{dt} = \frac{1}{2} \mu \omega^2 y_m^2 v^2 \cos^2(kx - \omega t) \]
while potential energy is given by the work done by tension,
\[ dU = F(dl - dx) = \frac{1}{2} Fdx \left(\frac{\partial y}{\partial x} \right)^2 \]
and thus
\[ \frac{dU}{dt} = \frac{1}{2} Fvk^2y_m^2\cos^2(kx - \omega t) \]
Note that $dK + dU$ is not constant - it is 0 at the peak, and they both increase on the way down. Energy is thus being transmitted. Moreover, substituting $F = \mu v^2$ and $kv = \omega$ into the second equation reveals that $dK = dU$. 

\subsubsection{Power and Intensity}
The power is given by 
\[ P = \frac{dE}{dt} = \frac{dK}{dt} + \frac{dU}{dt} = \mu \omega^2 y_m^2 v \cos^2(kx - \omega t) \]
Thus average power is 
\[ P_{av} = \frac{1}{2} \mu \omega^2 y_m^2 v \]
over a large number of cycles, assuming no dissipation. The dependence of average power on the square of amplitude and frequency is a general characteristic of waves.\bigskip

It is more effective to describe some wavefronts (e.g. spherical) in terms of intensity, not power, given by 
\[ I = \frac{P_{av}}{A} \]
where $A$ is the area of the wavefront. It is better to use intensity because while energy in each wavefront is the same, it is being spread over a larger area as it propagates. \bigskip

For spherical or circular waves, amplitude is not constant (but \textit{intensity}, not power!, is still proportional to its square). Thus, when you double the distance from a spherical light source, the intensity is multiplied by $1/4$ and the amplitude is multiplied by $1/2$. \bigskip

For cylindrical waves, the area of dissipation is the lateral surface area of the cylinder, which grows linearly with distance. Thus, intensity is inversely proportional to distance. \bigskip

Intensity can also be written in terms of the energy density, 
\[ I = uv \]
where $u$ is the energy density. 

\subsection{The Principle of Superposition}
The principle of superposition says that when several waves combine at a point, the resultant displacement is the sum of all the individual ones; that is, 
\[ y(x,t) = y_1(x,t) + y_2(x,t) + ...\]
and it holds true for mechanical waves in elastic media when the restoring force follows Hooke's law (that is, it varies linearly with displacement). 

\subsubsection{Fourier Analysis}
All periodic waves can be broken down into a linear combination of sinusoidal component waves of (potentially) different frequencies and amplitudes. \bigskip

For a waveform at a particular time with wavelength $\lambda$, the displacement is given by 
\[ y(x) = A_0 + A_1\sin{kx} + A_2\sin{2kx} + ... + B_1\cos{kx} + B_2\cos{2kx} + ... \]
A waveform may change its shape if the speeds of these component waves are different or if the medium is dissipative, causing energy loss. Dissipation is often related to speed, so it is waves at high particle speeds (high frequencies, since $u_y \propto f$) that are affected the most. 

\subsection{Interference of Waves}
Interference is when two waves combine at a point. Let the two waves be 
\[ y_1(x,t) = y_m \sin(kx - \omega t - \phi_1) \]
\[ y_2(x,t) = y_m \sin(kx - \omega t - \phi_2) \]
so that their interference produces a wave 
\[ y(x,t) = [2y_m\cos(\Delta \phi /2 )] \sin(kx - \omega t - \phi') \]
where $\phi' = (\phi_1+\phi_2)/2$ and $\Delta \phi$ is the phase difference. This result comes from the sum to product trigonometric formulae. When $\Delta \phi$ is 0, the waves undergo \textit{constructive interference}. When $\Delta \phi$ is $180^{\circ}$, the waves undergo \textit{destructive interference}. \bigskip

More generally, should the waves have differing amplitudes $y_{m1}$ and $y_{m2}$, this becomes 
\[ y(x,t) = \sqrt{y_{m1}^2 + y_{m2}^2 + 2y_{m1}y_{m2}\cos{\Delta \phi}} \sin(kx - \omega t - \phi') \]
where
\[ \phi' = \sin^{-1}\left[ \frac{y_{m1}\sin{\phi_1} + y_{m2}\sin{\phi_2}}{\sqrt{y_{m1}^2 + y_{m2}^2 + 2y_{m1}y_{m2}\cos{\Delta \phi}}} \right] \]

\subsection{Standing Waves}
When two waves of equal amplitude and frequency are moving in opposite directions, a standing wave is created. Let the two waves be 
\[ y_1(x,t) = y_m\sin(kx - \omega t) \]
\[ y_2(x,t) = y_m\sin(kx + \omega t) \]
so that their resultant is 
\[ y(x,t) = [2y_m \sin{kx}] \cos{\omega t} \]
which represents the general equation of a standing wave. Points of 0 displacement are called nodes, while points of maximum displacement are antinodes. Also, note that amplitude is a function of $x$ for a standing wave; it is not the same for all points!\bigskip

Therefore, the antinodes are at maximum displacement, 
\[ kx = \left(n+\frac{1}{2}\right)\pi \rightarrow x = \left(n+\frac{1}{2}\right) \frac{\lambda}{2} \]
and the nodes are at minimum displacement, 
\[ kx = n\pi \rightarrow x = n\frac{\lambda}{2} \]
for non-negative integers $n=0, \; 1, \; 2...$\bigskip

Nodes are always at rest, so energy cannot flow past them: energy remains standing in the string. 

\subsubsection{Standing Wave Ratio}
Consider an incident wave of amplitude $A_i$ as it is reflected from a boundary with amplitude $A_r$. The superposition of these creates a standing wave, with maximum and minimum resultant amplitude given by 
\[ \textrm{SWR} = \frac{A_i+A_r}{A_i-A_r} = \frac{A_{max}}{A_{min}} \]
and the percent reflection, defined as ratio of average power in reflected wave to that in the incident wave (times 100), is given by 
\[ \% \; \textrm{reflection} = \frac{(\textrm{SWR}-1)^2}{(\textrm{SWR}+1)^2} \times 100 \]

\subsubsection{Standing Waves on a String}
We can create standing waves by plucking a string. It may have a triangular shape when we release it, but the higher frequency terms from Fourier Analysis damp out quickly, leaving only the lowest possible frequency. By holding the nodes and plucking at the antinodes, we can create standing waves. \bigskip

For a string of length $L$, with $n$ loops, the wavelength is given by 
\[ L = n\frac{\lambda}{2} \rightarrow \lambda = \frac{2L}{n} \]
because the spacing between nodes is $\lambda/2$. Also, from $v = \lambda f$ we can find
\[ f = \frac{nv}{2L} \]
Why does this string have infinite frequencies of oscillation, but the block-spring system in SHM only has 1? Well, a lumped system of N elements has N different oscillating frequencies. While the block and spring have only one way to store potential and kinetic energy, the string has an infinite number of ways. 

\subsubsection{Resonance on Standing Waves}
When you shake a string, 
\begin{itemize}
    \item If the hand is out of phase ($\omega'' \neq \omega$), you do work on rope and rope does work on you
    \item If the hand is in phase ($\omega'' \approx \omega$), you do work on rope
\end{itemize}
The driving frequency is not exactly natural frequency at resonance due to damping. Also, the apparent nodes are not true nodes again because energy must be flowing past the nodes to counteract losses from damping.

\subsection{Wave Reflection}
\subsubsection{At a Boundary}
Consider a pulse reflecting off a fixed boundary. The rope exerts an upwards force on the wall as the pulse arrives, so the wall exerts an equal but opposite downwards force. Thus, the reflection is flipped upside down by this force, meaning the pulse changes phase by $180^{\circ}$.\bigskip

Consider a pulse reflecting off a free end. The pulse causes the free end to shoot up, exerting an upwards force just like it would if the pulse kept traveling. Thus, the reflected pulse is in phase with the incident one. 

\subsubsection{At a Change in Medium}
Consider two strings of different mass density, $\mu_1$ and $\mu_2$, tied together. This time, a wave that travels from 1 to 2 is partially transmitted and partially reflected. 
\begin{itemize}
    \item If $\mu_2 > \mu_1$: reflected wave is shifted $180^{\circ}$, $v_1 > v_2$, and $\lambda_1 > \lambda_2$.
    \item If $\mu_2 < \mu_1$: reflected wave stays in phase, $v_1 < v_2$, and $\lambda_1 < \lambda_2$. 
\end{itemize}
In both cases, energy is transmitted from string 1 to 2. However, the best way to minimize energy transfer is by making $\mu_2 << \mu_1$, in which case the boundary is very similar to the free end discussed above. Note that frequencies are the same to avoid discontinuity.

\textbf{All our discussion of transverse ways have so far lied in one plane; this is known as being \textit{plane polarized}.}

\end{document}