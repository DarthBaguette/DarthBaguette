\documentclass[../PhysicsFormulae.tex]{subfiles}
\begin{document}

\subsection{Simple Harmonic Motion}
Simple harmonic motion occurs for sinusoidal motion of constant frequency and amplitude. Most commonly, a block attached to a spring fixed at one end has \textit{equation of motion}
\[ \ddot{x} + \frac{k}{m}x = 0 \]
which has solution 
\[ x = x_m \cos(\omega t + \phi) \]
Note that the acceleration is related to position as $a = -\omega^2 x$, which is characteristic of SHM. The angular frequency, period, and frequency are given by
\[ \omega = \sqrt{\frac{k}{m}} \rightarrow T = 2\pi \sqrt{\frac{m}{k}} \rightarrow f = \frac{1}{2\pi}\sqrt{\frac{k}{m}} \]
while amplitude and phase constant $\phi$ depend on initial conditions. \bigskip

Energy considerations in our spring-block system yield
\[ U = \frac{1}{2}kx^2 = \frac{1}{2}kx_m^2\cos^2(\omega t + \phi) \]
\[ K = \frac{1}{2}mv^2 = \frac{1}{2}m\omega^2x_m^2\sin^2(\omega t + \phi) \]
and solving for velocity v yields
\[ v = \pm \sqrt{\frac{k}{m}(x_m^2 - x^2)} \]

\subsection{Applications of SHM}
\subsubsection{Simple Pendulum}
The equation of motion is approximately 
\[ \ddot{\theta} + \frac{g}{l}\theta = 0 \]
for small displacements. But the full solution has a period of
\[ T = 2\pi \sqrt{\frac{l}{g}} \left( 1 + \frac{1}{2^2}\sin^2{\frac{\theta_m}{2}} + \frac{3^2}{2^2\cdot 4^2}\sin^4{\frac{\theta_m}{2}} + ...\right) \]

\subsubsection{Torsional Pendulum}
A torsional pendulum is under restoring torque of $\uptau = -\kappa \theta$, so its equation of motion is
\[ \ddot{\theta} + \frac{\kappa}{I}\theta = 0 \]
which has solution 
\[ \theta = \theta_m\cos(\omega t + \phi) \]
provided that
\[ \omega = \sqrt{\frac{\kappa}{I}} \rightarrow T = 2\pi \sqrt{\frac{I}{\kappa}} \rightarrow f=\frac{1}{2\pi}\sqrt{\frac{\kappa}{I}} \]

\subsubsection{Foucault's Pendulum}
The period of Foucault's pendulum, which demonstrates Earth's rotation, is given by
\[ T = \frac{T_E}{\sin{L}} \]
where $T_E$ is 24 hours and $L$ is the latitude. 

\subsubsection{Vertical Spring Oscillation}
In a uniform gravitational field, the equation of motion for a vertical spring is
\[ \ddot{x} + \frac{k}{m}x - g = 0 \]
which has solution
\[ x = x_m \cos{\omega t + \phi} + \frac{mg}{k} \]
meaning that values of $\omega, f, T, v, a$ stay the same, but equilibrium shifts by $\frac{mg}{k}$. Moreover, changes in \textit{total} potential energy are the same. 

\subsection{Damped Harmonic Motion}
Consider a spring-mass system subjected to a damping force of $D=-bv$. The equation of motion is
\[ \ddot{x} + \frac{b}{m}\dot{x} + \frac{k}{m}x = 0 \]
which has solution
\[ x = x_m e^{-\frac{bt}{2m}}\cos(\omega't + \phi) \]
where
\[ \omega' = \sqrt{\frac{k}{m} - \left( \frac{b}{2m} \right)^2} \]
which means that when $b = 2\sqrt{km}$, we have $\omega' = 0$. This is \textit{critical damping} and it is used to minimize unwanted vibrations so that they decay to 0 as fast as possible.

\subsection{Forced Oscillations and Resonance} 
We can force a system to oscillate at a driving angular frequency $\omega''$ instead of its natural one $\omega$. The equation of motion is
\[ \ddot{x} + \frac{b}{m}\dot{x} + \frac{k}{m}x = \frac{F_m}{m} \cos(\omega''t) \]
which has solution
\[ x = \frac{F_m}{m} \cos(\omega''t - \beta) \]
where
\[ G = \sqrt{m^2(\omega''^2-\omega^2)^2 + b^2 \omega''^2} \]
and 
\[ \beta = \sin^{-1}\left(\frac{b\omega''}{G}\right) \]
When $\omega'' = \omega$ we have resonance, which gives the maximum displacement. Without damping, this would be infinitely large. 

\subsection{Two-Body Oscillations}
Suppose masses of $m_2$ and $m_1$ are connected by a spring, with the former on the left and positive rightwards. The stretch is then $x=x_1-x_2-l$, and Newton's second law yields
\[ m_1\ddot{x}_1 = -kx \]
\[ m_2\ddot{x}_2 = kx \]
Linearly combining with weights of $m_2$ and $-m_1$, respectively, yields
\[ \frac{m_1m_2}{m_1+m_2}(\ddot{x}_1 - \ddot{x}_2) = -kx \]
where the reduced mass is 
\[ m = \frac{m_1m_2}{m_1+m_2} \]
With $\ddot{x}=\ddot{x}_1-\ddot{x}_2$ this becomes
\[ \ddot{x} + \frac{k}{m}x = 0 \]
and it can also be shown that the kinetic energy is given by 
\[ K = \frac{1}{2}mv^2 \]
where $m$ is reduced mass and $v=v_1-v_2$ is relative velocity.

\subsection{Springs}
\subsubsection{Parallel Springs}
For a parallel spring combination, the effective spring constant is
\[ k = k_1 + k_2 + ... + k_n\]

\subsubsection{Series Springs}
For a series spring combination the effective spring constant is 
\[ \frac{1}{k} = \frac{1}{k_1} + \frac{1}{k_2} +...+ \frac{1}{k_n} \]

\subsubsection{Delta-Star}

To go from Delta (triangular) to Star configuration, apply
\[k_a = \cfrac{k_{AB}k_{BC} + k_{BC}k_{CA} + k_{CA}k_{AB}}{k_{BC}}\]
\[k_b = \cfrac{k_{AB}k_{BC} + k_{BC}k_{CA} + k_{CA}k_{AB}}{k_{CA}}\]
\[k_c = \cfrac{k_{AB}k_{BC} + k_{BC}k_{CA} + k_{CA}k_{AB}}{k_{AB}}\]
In the special case $k_{AB}=k_{BC}=k_{CA}=k$ then $k_a=k_b=k_c=3k$.
More generally, we have
\[k_{star} = \frac{1}{k_{opp}}\sum k_{pairs} \]

\subsubsection{Star-Delta}
To go from the Star to Delta (Triangular) Configuration, apply
\[k_{AB} = \cfrac{k_a k_b}{k_a + k_b + k_c}\]
\[k_{BC} = \cfrac{k_b k_c}{k_a + k_b + k_c}\]
\[k_{CA} = \cfrac{k_c k_a}{k_a + k_b + k_c}\]
More generally, we have
\[k_{delta} = \cfrac{\prod k_{adj}}{\sum k_{star}} \]

\end{document}