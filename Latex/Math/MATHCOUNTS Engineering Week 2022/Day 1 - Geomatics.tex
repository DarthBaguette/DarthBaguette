\documentclass{article}
\usepackage[utf8]{inputenc}
\usepackage{amsmath}
\usepackage[margin=1in]{geometry}
\setlength{\parindent}{0pt}

\title{Geomatics Engineering Problems}
\author{Joshua Lin}
\date{February 21, 2022}

\begin{document}
\maketitle

\section{Rockstone Monument}

\textbf{1.1} In Rockstone City, no building is permitted to be taller than the monument to the city’s first mayor, Mayor Rockstone. A new developer wants to build an apartment building in Rockstone City, but does not know how tall the monument to Mayor Rockstone is. A 35-foot telephone pole next to the monument casts a shadow that is 15 feet long. If the monument’s shadow is 257 feet long at the same time of day, how tall is the monument to Mayor Rockstone? Express your answer to the nearest whole number of feet. \bigskip

\textbf{Solution} The ratio of $\mathrm{height/shadow}$ is identical for all buildings. Thus, 
\[ \mathrm{ \frac{35 \; ft}{15 \; ft} } = \frac{h}{257 \; \mathrm{ft}} \]
To the nearest whole foot, 
\[ \boxed{ h = 600 \; \mathrm{ft} } \]

\section{LiDAR}

\textbf{1.2} LiDAR, or Light Detection and Ranging, is a method used in geomatics engineering to determine distances and other measurements of natural formations either on land or underwater. To measure distance using LiDAR, a laser light source is shot from one point to another and bounces back. The distance d, in feet, between the two points is calculated by multiplying the time t, in seconds, it takes the laser light to travel from one point to another and back by the speed of light c, in feet per second, and then dividing the result by 2. Since the light travels at a constant speed of 983,571,056 ft/s, we can use the following formula:

\[ d = \frac{t \times 983,571,056}{2} \] 

The developer uses LiDAR to determine the length of the rectangular plot of land on which she plans to build the apartment building. If the time it took the laser light to travel the length of the plot and back was $4.03 \times 10^{-7}$ seconds, how long is the plot of land? Express your answer to the nearest whole number of feet. \bigskip

\textbf{Solution} Substituting, 
\[ d = \frac{4.03 \times 10^{-7} \times 983,571,056}{2} \]
\[ \boxed{d = 198 \; \mathrm{ft}} \]

\section{Problem 3}

\textbf{1.3} The rectangular plot of land, which has a width of 110 feet, has a grade (or slope) of 30\%, which means over 100 feet, the land rises 30 feet. Assuming the land rises across the width of the plot, how many total feet does the land rise? 
\[(110 \; \mathrm{ft} ) ( 0.3) = \boxed{33 \; \mathrm{ft}}\]
\bigskip

\section{Problem 4}

\textbf{1.4} Before beginning construction, the ground must be leveled in order to build on a flat surface. If the company contracted by the developer charges \$125 per cubic yard for dirt removal, what will be the total charge to remove 135,000 $\mathrm{ft^3}$ of dirt to level this plot of land? \bigskip 

\textbf{Solution} Using dimensional analysis, 
\[ C = \mathrm{(135,000 \; ft^3)\left(\frac{1 \; yd}{3 \; ft}\right)^3\left(\frac{\$125}{yd^3}\right)}\]
\[ = \boxed{\$625,000 }\]

\end{document}